\begin{document}
\begin{equation}
	V(s) = R I(s) + sL I(s) + \dfrac{1}{sC} I(s)
\end{equation}

\begin{equation}
	\Rightarrow V(s) = I(s)\left(R + Ls + \dfrac{1}{sC}\right)
\end{equation}

\begin{equation}
	\Rightarrow I(s) = \dfrac{V(s)}{\left(R + Ls + \dfrac{1}{sC}\right)} \label{eq: 4}
\end{equation}

The transfer function \(H(s)\) for a series RLC circuit is given by:

\begin{align}
	H(s) &= \frac{V_{\text{out}}(s)}{V_{\text{in}}(s)} 
\end{align}
Express H(s) in terms of s :
\begin{align}
	H(s) &= \frac{1}{sRC + 1 + j(sL - \frac{1}{sC})}
\end{align}

Replacing $\text{s}$ with $j\omega$ :

\begin{align}
	H(j\omega) = \frac{1}{1 - \omega^2 LC + j(\omega RC - \frac{1}{\omega C})}
\end{align}
The absolute value of 
\begin{align}
	H(j\omega)| = \left| \frac{1}{1 - \omega^2 LC + j(\omega RC - \frac{1}{\omega C})} \right|
\end{align}

This is the expression for \(H(j\omega)\) for a series RLC circuit. Below is the graph of absolute value of $H(j\omega)$ :
\end{document}
